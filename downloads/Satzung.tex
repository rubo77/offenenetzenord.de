\documentclass[12pt,a4paper]{article}
\usepackage[utf8]{inputenc}
 \usepackage[ngerman]{babel}
 \usepackage[T1]{fontenc}
\usepackage{amsmath}
\usepackage{amsfonts}
\usepackage{amssymb}
\usepackage{geometry}
\usepackage{fancyhdr}
\setlength{\headheight}{15pt}
%Package Fancyhdr Warning: \headheight is too small (12.0pt):
% Make it at least 14.49998pt.
% We now make it that large for the rest of the document.
% This may cause the page layout to be inconsistent, however.%

%======================================================
% Config der Satzung, Leerzeichen am Ende sind wichtig!
%======================================================
\newcommand{\Verein}{Freifunk Nord} %Name des Vereins
\newcommand{\Gdatum}{07.10.2017} %Gründungsdatum
\newcommand{\Ort}{Kiel} %Sitz des Vereins
\newcommand{\VorstandEins}{Ruben Barkow}
\newcommand{\VorstandZwei}{Felix von Courten}
%======================================================


%======================================================
% Ab hier muss fast nichts mehr geändert werden
%======================================================

\title{Satzung {\Verein} e.V.}

\begin{document}
\pagestyle{fancy}
\chead{Satzung {\Verein} e.V.}
\begin{center}
\section*{Satzung \\ {\Verein} e.V.}
in der Entwurfsfassung vom 13.06.2019
\end{center}
\bigskip
\subsection*{§\,1 Name und Sitz des Vereins}
\begin{enumerate}
\item Der Verein ist in das Vereinsregister eingetragen und trägt den Namen\\
„{\Verein} e.V.“.
\item Der Verein hat seinen Sitz in {\Ort}.
\item Das Geschäftsjahr des Vereins ist das Kalenderjahr.
\end{enumerate}

\subsection*{§\,2 Zweck des Vereins, Gemeinnützigkeit, Auflösung und Vermögen}
\begin{enumerate}
\item Zweck des Vereins ist die Förderung der Volks- und Berufsbildung, Förderung von Wissenschaft und Forschung, die Förderung internationaler Gesinnung, der Toleranz auf allen Gebieten der Kultur und des Völkerverständigungsgedankens, die allgemeine Förderung des demokratischen Staatswesens in der Bundesrepublik Deutschland sowie die Förderung des bürgerschaftlichen Engagements bezüglich dieser Zwecke.
\item Hierzu fördert der Verein ideell, materiell und finanziell insbesondere:
\begin{enumerate}
\item den Aufbau und Betrieb kabelloser und kabelgebundener Computernetzwerke, die der Allgemeinheit frei zugänglich sind,
\item die Erforschung, Anwendung und Verbreitung freier Netz\-werktechnologien,
\item die Verbreitung und Vermittlung von Wissen über Funk- und Netz\-werktechnologien,
\item den Zugang zu Informationstechnologie für sozial benachteiligte Personen,
\item die Schaffung experimenteller Kommunikations- und Infrastrukturen sowie Bürgerdatennetze,
\item kulturelle, technologische und soziale Bildungs- und Forschungsprojekte,
\item die Fort- und Weiterbildung im Rahmen der Informations- und Kommunikationstechnologie,
\item das Zusammenwirken mit öffentlichen und privaten Bildungseinrichtungen (Universitäten, Hochschulen, Volkshochschulen etc.),
\item die Förderung der nationalen und internationalen Zusammenarbeit auf dem Gebiet der Informations- und Kommunikationstechnologie,
\end{enumerate}
\item Der Verein ist frei und unabhängig. Er verfolgt ausschließlich und unmittelbar gemeinnützige Zwecke im Sinne des Abschnitts „Steuerbegünstigte Zwecke“ der Abgabenordnung. Er ist selbstlos tätig und verfolgt nicht in erster Linie eigenwirtschaftliche Zwecke. Die Mittel des Vereins dürfen nur für satzungsgemäße Zwecke verwendet werden. Es darf keine Person durch Ausgaben, die dem Vereinszweck fremd sind, oder durch unverhältnis hohe Vergütungen begünstigt werden. Die Mitglieder erhalten keine Zuwendungen aus den Mitteln des Vereins.
\item Bei Auflösung der Körperschaft oder bei Wegfall steuerbegünstigter Zwecke fällt das Vermögen des Vereins an die Toppoint e.V. oder an eine andere steuerbegünstigte Körperschaft oder Körperschaft öffentlichen Rechts mit vergleichbarer Zielsetzung, welche es unmittelbar für gemeinnützige Zwecke verwenden darf. 
\item Ausscheidende Mitglieder haben keinen Anspruch auf das Vereinsvermögen. 
\item Über die Auflösung des Vereines entscheidet eine Mitgliederversammlung, die eigens zu diesem Zweck einberufen wird. Die Auflösung gilt als beschlossen wenn dreiviertel der abgegebenen Stimmen dafür stimmen. 
\end{enumerate}

\subsection*{§\,3 Mitgliedschaft}
\begin{enumerate}
\item Mitglieder können natürliche und juristische Personen werden, die gewillt sind, die gemeinnützigen Ziele des Vereins zu fördern und diesen in der Durchführung seiner Aufgaben zu unterstützen. Bei Minderjährigen ist die Zustimmung des gesetzlichen Vertreters erforderlich.
%\item Der Aufnahmeantrag ist in Textform an den Vorstand zu richten, der über die Aufnahme des Antragstellers entscheidet. Der Aufnahmeantrag muß eine persönlich zugeordnete Kommunikationsadresse enthalten. 
\item Der Aufnahmeantrag ist in Textform an den Vorstand zu richten, der über die Aufnahme des Antragstellers entscheidet. Der Aufnahmeantrag muß eine persönlich zugeordnete Identifikationsadresse für die von der Mitgliederversammlung festgelegten Kommunikationsverfahren gemäß § 4 Abs. 8 (h) enthalten.
\item Der Beitritt gilt erst dann als vollzogen, wenn der erste Mitgliedsbeitrag entrichtet worden ist.
\item Die Mitglieder haben das Recht, an der Mitgliederversammlung des Vereins teilzunehmen, Anträge zu stellen, und das Stimmrecht auszuüben. Juristische Personen üben ihr Stimmrecht durch bevollmächtigte Vertreter aus. Das aktive Stimmrecht besitzen Mitglieder mit Erreichen des 16. Lebensjahrs. Das passive Wahlrecht beginnt mit Erreichen des 18. Lebensjahrs.
\item Jedes Mitglied hat einen Jahresbeitrag zu leisten, dessen Höhe und Fälligkeit von der Mitgliederversammlung beschlossen wird.
\item Der Vorstand kann der Mitgliederversammlung die Ernennung von Ehrenmitgliedern vorschlagen. Ehrenmitglieder sind von Beitragszahlungen freigestellt und haben auf Mitgliederversammlungen volles Stimmrecht.
\item Im begründeten Einzelfall kann für ein Mitglied durch Vorstandsbeschluss ein von der Beitragsordnung abweichender Beitrag festgesetzt werden.
\item Im Falle nicht fristgerechter Entrichtung der Beiträge ruht die Mitgliedschaft. Das Mitglied trägt die in der Beitragsordnung festgelegten Mahngebühren zusätzlich zum Beitrag.
\item Die Mitgliedschaft endet durch Austritt, Ausschluss oder Tod.
\item Der Austritt muss durch Mitteilung in Textform an den Vorstand erklärt werden. Er wird mit Endes des Geschäftsjahrs wirksam und muss sechs Wochen vor dessen Ablauf mitgeteilt worden sein.
\item Ein Ausschluss kann nur aus wichtigem Grund erfolgen. Wichtige Gründe sind insbesondere ein die Vereininteressen schädigendes Verhalten, die Verletzung satzungsmäßiger Pflichten oder Beitragsrückstände von mindestens einem Jahr. Über den Ausschluss entscheidet der Vorstand. Dem Mitglied ist vor dem Ausschlussbeschluss Gehör zu gewähren.
\item Der Ausgeschlossene kann innerhalb eines Monats nach Zugang des Beschlusses Einspruch einlegen und die nächste Mitgliederversammlung anrufen, von der die Gültigkeit des Ausschlusses mit Zweidrittelmehrheit der anwesenden Mitglieder bestätigt wird. Vom Zeitpunkt des Vorstandsbeschlusses bis zur Entscheidung über den Ausschluss ruht die Mitgliedschaft.
\item Fördermitgliedschaften sind möglich. Fördermitglied des Vereins kann jede Person werden, die sich mit den Zielen des Vereins verbunden fühlt und den Verein finanziell und ideell unterstützen will. Fördermitglieder besitzen kein Stimmrecht.
\end{enumerate}

\subsection*{§\,4 Die Mitgliederversammlung}
\begin{enumerate}
%\item Die ordentliche Mitgliederversammlung findet einmal jährlich auf Einladung eines Vorstandsmitgliedes am Sitz des Vereins statt.
\item Die ordentliche Mitgliederversammlung findet einmal jährlich auf Einladung eines Vorstandsmitgliedes entweder in geeigneter virtueller Form oder am Sitz des Vereins statt.
\item Eine außerordentliche Mitgliederversammlung ist unverzüglich und unter genauer Angabe von Gründen einzuberufen, wenn es das Interesse des Vereins erfordert oder wenn mindestens 25\% der Mitglieder dies schriftlich unter Angabe des Zwecks und der Gründe vom Vorstand verlangen.
\item Die Mitgliederversammlung bestimmt einen Versammlungsleiter und einen Protokollführer.
\item Die Beschlüsse der Mitgliederversammlung werden in einem Protokoll niedergelegt und mit den Unterschriften des Versammlungsleiters und des Protokollführers beurkundet.
\item Die Einladung zur Mitgliederversammlung ist den Mitgliedern in Textform mindestens vier Wochen vorher zu übersenden, wobei die Einladung als bewirkt gilt, wenn sie fristgerecht an die letzte vom Mitglied angegebene Adresse abgesandt worden ist.
%\item Mitglieder können sich durch einen Bevollmächtigten vertreten lassen. Die Vertretungsbefugnis ist dem Versammlungsleiter schriftlich nachzuweisen. Kein Bevollmächtigter kann mehr als ein übertragenes Stimmrecht ausüben.
\item Der Vorstand kann beschließen, die Mitgliederversammlung in geeigneter virtueller Form (Text- oder Voicechat) durchzuführen. Beschlüsse werden dann in Textform oder fernmündlich unter Beachtung der in Abs. 8 (h) festgelegten Verfahrensordnung %gefasst. Wenn 25\% der Mitglieder innerhalb von zwei Wochen nach Versand der Einladung der virtuellen Form widersprechen, muß die Mitgliederversammlung als Präsenzveranstaltung am Sitz des Vereins abgehalten werden.
\item In einer Präsenzveranstaltung können sich Mitglieder durch einen Bevollmächtigten vertreten lassen. Die Vertretungsbefugnis ist dem Versammlungsleiter schriftlich nachzuweisen. Kein Bevollmächtigter kann mehr als ein übertragenes Stimmrecht ausüben.
\item Der Mitgliederversammlung obliegen insbesondere:
\begin{enumerate}
\item die Beschlussfassung über alle den Verein betreffenden Angelegenheiten von grundsätzlicher Bedeutung
\item die Entscheidung über zwei Wochen vor Beginn der Mitgliederversammlung beim Vorstand eingegangene Anträge
\item die Entgegennahme des Jahresberichtes des Vorstands
\item die Entlastung des Vorstands
\item die Wahl der Vorstandsmitglieder
\item die Beschlussfassung über Satzungsänderungen, mit Dreiviertelmehrheit auch über Änderungen des Vereinszwecks,
\item die Festsetzung der Mitgliedsbeiträge,
\item die Festsetzung der Verfahren und Kommunikationsmittel für virtuelle Versammlungen und Abstimmungen, 
\item die Auflösung des Vereins gemäß §\,2, Ziffer 4 und 6 dieser Satzung.
\end{enumerate}
\end{enumerate}
\subsection*{§\,5 Der Vorstand}
\begin{enumerate}
\item Der Vorstand des Vereins besteht aus mindestens zwei Personen: dem Vorsitzenden und dem Schatzmeister. Jeder von ihnen vertritt den Verein gericht\-lich und außergerichtlich allein. Es können weitere Vorstandsmitglieder gewählt werden. Die gewählten Vorstandsmitglieder sind Vorstand im Sinne des §\,26 des Bürgerlichen Gesetzbuches.
\item Der Vorstand wird auf die Dauer von jeweils zwei Jahren gewählt. Nach Ablauf dieser Zeit bleibt er bis zur Wahl eines neues Vorstands kommissarisch im Amt. Scheidet ein Vorstandsmitglied während der Amtszeit aus, so können die übrigen Vorstandsmitglieder ein Ersatzmitglied berufen, der der Bestätigung durch die nächste Mitgliederversammlung bedarf.
\item Die Vorstandsmitglieder üben ihr Amt ehrenamtlich aus.
\item Dem Vorstand obliegen die laufende Geschäftsführung, die Ausführung der Beschlüsse der Mitgliederversammlung und die Verwaltung des Vereinsvermögens.
\item Beschlüsse des Vorstands werden mit einfacher Mehrheit gefasst. Bei gleicher Stimmzahl wird die Stimme des Schatzmeisters doppelt gezählt.
\item Der Vorstand kann zur Unterstützung und Wahrnehmung seiner Aufgaben Vereinsmitglieder berufen, die entweder auf Dauer oder nur zur Erfüllung einer zeitlich begrenzten Tätigkeit Funktionen übernehmen.
\item Mitglieder des Vorstands und nach Abs. 6 beauftragte Mitglieder haften dem Verein nur für grob fahrlässige und vorsätzliche Schädigung. 
\end{enumerate}

\end{document}
