\documentclass[a4paper,8pt]{article}
\usepackage[utf8]{inputenc}
\usepackage[T1]{fontenc}
\usepackage{eurosym}
\usepackage{geometry}
\geometry{a4paper,left=20mm,right=30mm,top=15mm,bottom=20mm}
\usepackage{fancybox}
\usepackage{multirow}
\setlength\parindent{0pt}

\begin{document}

\shadowbox{\bf \large{Mitglieds-/Nutzerantrag für natürliche Personen Offene Netze Nord e.V.}}
\ \\

Sehr geehrtes Mitglied!\\

Ich,\\
\begin{tabular}{||c|p{10cm}||}
\hline
\hline
Vorname &\\
\hline
Nachname &\\
\hline
Straße+Hausnummer &\\
\hline
PLZ &\\
\hline
Wohnort &\\
\hline
Chatname &\\
\hline
Telefon$^\star$ &\\
\hline
alternative & \multirow{2}{*}{} \\
Mailadresse$^\star$ &\\
\hline
Geburtsdatum$^\star$ &\\
\hline
\hline
\multicolumn{2}{r}{\tiny $^\star$ freiwillige Angabe}
\end{tabular}
\\
möchte Nutzer des Offene Netze Nord e.V. werden.\\

Ggf. bitte ankreuzen:
\begin{description}
\item[O] Ich möchte regelmässig aktiv im Verein Offene Netze Nord e.V. mitarbeiten und beantrage daher die Mitgliedschaft im Verein.
\end{description}

Ich bin damit einverstanden, dass meine Daten gespeichert und für Vereinszwecke genutzt werden.
Eine Weitergabe an Dritte ist im Rahmen der technischen und persönlichen Möglichkeiten ausgeschlossen.
\\

Es wird eine Aufnahmegebühr in der Höhe eines Jahresbeitrags fällig.
Die Höhe der Beiträge ist der jeweils gültigen Gebührenordnung zu entnehmen,
aktuell sind dies für ein normales Mitglied 35\euro{} und für Juristische Personen 70\euro{} pro Jahr.
\\

Beiträge sind am 1. Januar für das jeweilige Jahr im Voraus fällig.
\\

Es ist erforderlich, Beiträge per Dauerauftrag jährlich, im voraus auf das Vereinskonto bei der Fidor Bank
(IBAN: DE81 7002 2200 0020 1472 37, BIC: FDDODEMMXXX) zu ent\-richten.
Zur eindeutigen Zuordnung der Zahlung ist es sinnvoll, im Betreff den vollen Namen des Mitglieds anzugeben.
\\

Andere Zahlungsweisen können wir leider nicht berücksichtigen.
\\


\begin{tabular}{l r}
&\\
\hline
Ort, Datum & Unterschrift (bei Minderjährigen auch Unterschrift eines Erziehungsberechtigten)\\
&\\
\end{tabular}

\end{document}
